\section{Tips}

\subsection{定理}
\subsubsection{双対定理}
\[ \min\{ \bm{c}^\top \bm{x} \, | \, \bm{A} \bm{x} = \bm{b}, \bm{x} \geq \bm{0} \}
= \max\{ \bm{b}^\top \bm{w} \, | \, \bm{A}^\top \bm{w} \leq \bm{c} \} \]
\[ \min\{ \bm{c}^\top \bm{x} \, | \, \bm{A} \bm{x} \geq \bm{b}, \bm{x} \geq \bm{0} \}
= \max\{ \bm{b}^\top \bm{w} \, | \, \bm{A}^\top \bm{w} \leq \bm{c}, w \geq \bm{0} \} \]

\subsubsection{ヤング図形}

\subsection{STL-Algorithm}
\subsubsection{header: algorithm}
\begin{lstlisting}[caption=辞書順比較]
lexicographical_compare(begin(a),end(a),begin(b));
\end{lstlisting}
aとbを辞書順比較する。第四引数に比較関数を渡せる。
\begin{lstlisting}[caption=最大値、最小値]
max_element(begin(a),end(a));
min_element(begin(a),end(a));
minmax_element(begin(a),end(a));
\end{lstlisting}
最大値のイテレータ、最小値のイテレータ、そのペアをそれぞれ返す。第三引数に比較関数を渡せる。
\begin{lstlisting}[caption=順列]
is_permutation(begin(a),end(a),begin(b));
\end{lstlisting}
aがbの並べ替えになっていればtrueを返す。長さが違うことがある場合は第4引数にend(b)を渡す。最後の引数に等号比較関数を渡せる。
\begin{lstlisting}[caption=ソート済みコンテナ操作]
merge(begin(a),end(a),begin(b),end(b),begin(c));
inplace_merge(begin(a),middle,end(a),begin(b));
includes(begin(a),end(a),begin(b),end(b));
set_difference(begin(a),end(a),begin(b),end(b),begin(c));
set_intersection(begin(a),end(a),begin(b),end(b),begin(c));
set_symmetric_difference(begin(a),end(a),begin(b),end(b),begin(c));
\end{lstlisting}
\subsubsection{header: numeric}
\begin{lstlisting}[caption=和]
accumlate(begin(a),end(a),0);
\end{lstlisting}
第三引数に初期値、第四引数に演算を渡せる
\begin{lstlisting}[caption=累積和]
partial_sum(begin(a),end(a),begin(s)+1);
\end{lstlisting}
sにはaの累積和が入る。sはaより1長いこと。
\begin{lstlisting}[caption=階差数列]
adjacent_difference(begin(a),end(a),begin(s));
\end{lstlisting}
\begin{lstlisting}[caption=内積]
inner_product(begin(a),end(a),begin(s),0);
\end{lstlisting}
第四引数に初期値を渡す。第五引数に加法、第六引数に乗法に相当する演算を渡してやることもできる
\begin{lstlisting}[caption=連続した数列]
iota(begin(a),end(a),0);
\end{lstlisting}
この例だとaは[0,1,2,\ldots]となる
